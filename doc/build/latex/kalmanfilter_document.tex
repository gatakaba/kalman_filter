% Generated by Sphinx.
\def\sphinxdocclass{report}
\documentclass[letterpaper,10pt,dvipdfmx]{sphinxmanual}
\usepackage[utf8]{inputenc}
\DeclareUnicodeCharacter{00A0}{\nobreakspace}
\usepackage{cmap}
\usepackage[T1]{fontenc}

\usepackage{times}

\usepackage{longtable}
\usepackage{sphinx}
\usepackage{multirow}
\usepackage{eqparbox}
\usepackage{amsfonts}

\renewcommand{\figurename}{図 }
\renewcommand{\tablename}{TABLE }
\SetupFloatingEnvironment{literal-block}{name=LIST }



\title{kalmanfilter\_document Documentation}
\date{2016 年 08 月 09 日}
\release{1.0.0}
\author{gatakaba}
\newcommand{\sphinxlogo}{}
\renewcommand{\releasename}{リリース}
\setcounter{tocdepth}{0}
\makeindex

\makeatletter
\def\PYG@reset{\let\PYG@it=\relax \let\PYG@bf=\relax%
    \let\PYG@ul=\relax \let\PYG@tc=\relax%
    \let\PYG@bc=\relax \let\PYG@ff=\relax}
\def\PYG@tok#1{\csname PYG@tok@#1\endcsname}
\def\PYG@toks#1+{\ifx\relax#1\empty\else%
    \PYG@tok{#1}\expandafter\PYG@toks\fi}
\def\PYG@do#1{\PYG@bc{\PYG@tc{\PYG@ul{%
    \PYG@it{\PYG@bf{\PYG@ff{#1}}}}}}}
\def\PYG#1#2{\PYG@reset\PYG@toks#1+\relax+\PYG@do{#2}}

\expandafter\def\csname PYG@tok@ni\endcsname{\let\PYG@bf=\textbf\def\PYG@tc##1{\textcolor[rgb]{0.84,0.33,0.22}{##1}}}
\expandafter\def\csname PYG@tok@ow\endcsname{\let\PYG@bf=\textbf\def\PYG@tc##1{\textcolor[rgb]{0.00,0.44,0.13}{##1}}}
\expandafter\def\csname PYG@tok@vc\endcsname{\def\PYG@tc##1{\textcolor[rgb]{0.73,0.38,0.84}{##1}}}
\expandafter\def\csname PYG@tok@mh\endcsname{\def\PYG@tc##1{\textcolor[rgb]{0.13,0.50,0.31}{##1}}}
\expandafter\def\csname PYG@tok@bp\endcsname{\def\PYG@tc##1{\textcolor[rgb]{0.00,0.44,0.13}{##1}}}
\expandafter\def\csname PYG@tok@se\endcsname{\let\PYG@bf=\textbf\def\PYG@tc##1{\textcolor[rgb]{0.25,0.44,0.63}{##1}}}
\expandafter\def\csname PYG@tok@gr\endcsname{\def\PYG@tc##1{\textcolor[rgb]{1.00,0.00,0.00}{##1}}}
\expandafter\def\csname PYG@tok@kc\endcsname{\let\PYG@bf=\textbf\def\PYG@tc##1{\textcolor[rgb]{0.00,0.44,0.13}{##1}}}
\expandafter\def\csname PYG@tok@nb\endcsname{\def\PYG@tc##1{\textcolor[rgb]{0.00,0.44,0.13}{##1}}}
\expandafter\def\csname PYG@tok@kt\endcsname{\def\PYG@tc##1{\textcolor[rgb]{0.56,0.13,0.00}{##1}}}
\expandafter\def\csname PYG@tok@m\endcsname{\def\PYG@tc##1{\textcolor[rgb]{0.13,0.50,0.31}{##1}}}
\expandafter\def\csname PYG@tok@cm\endcsname{\let\PYG@it=\textit\def\PYG@tc##1{\textcolor[rgb]{0.25,0.50,0.56}{##1}}}
\expandafter\def\csname PYG@tok@vi\endcsname{\def\PYG@tc##1{\textcolor[rgb]{0.73,0.38,0.84}{##1}}}
\expandafter\def\csname PYG@tok@go\endcsname{\def\PYG@tc##1{\textcolor[rgb]{0.20,0.20,0.20}{##1}}}
\expandafter\def\csname PYG@tok@il\endcsname{\def\PYG@tc##1{\textcolor[rgb]{0.13,0.50,0.31}{##1}}}
\expandafter\def\csname PYG@tok@kn\endcsname{\let\PYG@bf=\textbf\def\PYG@tc##1{\textcolor[rgb]{0.00,0.44,0.13}{##1}}}
\expandafter\def\csname PYG@tok@w\endcsname{\def\PYG@tc##1{\textcolor[rgb]{0.73,0.73,0.73}{##1}}}
\expandafter\def\csname PYG@tok@o\endcsname{\def\PYG@tc##1{\textcolor[rgb]{0.40,0.40,0.40}{##1}}}
\expandafter\def\csname PYG@tok@s1\endcsname{\def\PYG@tc##1{\textcolor[rgb]{0.25,0.44,0.63}{##1}}}
\expandafter\def\csname PYG@tok@err\endcsname{\def\PYG@bc##1{\setlength{\fboxsep}{0pt}\fcolorbox[rgb]{1.00,0.00,0.00}{1,1,1}{\strut ##1}}}
\expandafter\def\csname PYG@tok@kr\endcsname{\let\PYG@bf=\textbf\def\PYG@tc##1{\textcolor[rgb]{0.00,0.44,0.13}{##1}}}
\expandafter\def\csname PYG@tok@gi\endcsname{\def\PYG@tc##1{\textcolor[rgb]{0.00,0.63,0.00}{##1}}}
\expandafter\def\csname PYG@tok@c\endcsname{\let\PYG@it=\textit\def\PYG@tc##1{\textcolor[rgb]{0.25,0.50,0.56}{##1}}}
\expandafter\def\csname PYG@tok@kd\endcsname{\let\PYG@bf=\textbf\def\PYG@tc##1{\textcolor[rgb]{0.00,0.44,0.13}{##1}}}
\expandafter\def\csname PYG@tok@nc\endcsname{\let\PYG@bf=\textbf\def\PYG@tc##1{\textcolor[rgb]{0.05,0.52,0.71}{##1}}}
\expandafter\def\csname PYG@tok@gh\endcsname{\let\PYG@bf=\textbf\def\PYG@tc##1{\textcolor[rgb]{0.00,0.00,0.50}{##1}}}
\expandafter\def\csname PYG@tok@cp\endcsname{\def\PYG@tc##1{\textcolor[rgb]{0.00,0.44,0.13}{##1}}}
\expandafter\def\csname PYG@tok@nn\endcsname{\let\PYG@bf=\textbf\def\PYG@tc##1{\textcolor[rgb]{0.05,0.52,0.71}{##1}}}
\expandafter\def\csname PYG@tok@mi\endcsname{\def\PYG@tc##1{\textcolor[rgb]{0.13,0.50,0.31}{##1}}}
\expandafter\def\csname PYG@tok@nl\endcsname{\let\PYG@bf=\textbf\def\PYG@tc##1{\textcolor[rgb]{0.00,0.13,0.44}{##1}}}
\expandafter\def\csname PYG@tok@mb\endcsname{\def\PYG@tc##1{\textcolor[rgb]{0.13,0.50,0.31}{##1}}}
\expandafter\def\csname PYG@tok@sc\endcsname{\def\PYG@tc##1{\textcolor[rgb]{0.25,0.44,0.63}{##1}}}
\expandafter\def\csname PYG@tok@sd\endcsname{\let\PYG@it=\textit\def\PYG@tc##1{\textcolor[rgb]{0.25,0.44,0.63}{##1}}}
\expandafter\def\csname PYG@tok@gd\endcsname{\def\PYG@tc##1{\textcolor[rgb]{0.63,0.00,0.00}{##1}}}
\expandafter\def\csname PYG@tok@gt\endcsname{\def\PYG@tc##1{\textcolor[rgb]{0.00,0.27,0.87}{##1}}}
\expandafter\def\csname PYG@tok@sr\endcsname{\def\PYG@tc##1{\textcolor[rgb]{0.14,0.33,0.53}{##1}}}
\expandafter\def\csname PYG@tok@mo\endcsname{\def\PYG@tc##1{\textcolor[rgb]{0.13,0.50,0.31}{##1}}}
\expandafter\def\csname PYG@tok@sb\endcsname{\def\PYG@tc##1{\textcolor[rgb]{0.25,0.44,0.63}{##1}}}
\expandafter\def\csname PYG@tok@c1\endcsname{\let\PYG@it=\textit\def\PYG@tc##1{\textcolor[rgb]{0.25,0.50,0.56}{##1}}}
\expandafter\def\csname PYG@tok@ne\endcsname{\def\PYG@tc##1{\textcolor[rgb]{0.00,0.44,0.13}{##1}}}
\expandafter\def\csname PYG@tok@no\endcsname{\def\PYG@tc##1{\textcolor[rgb]{0.38,0.68,0.84}{##1}}}
\expandafter\def\csname PYG@tok@nv\endcsname{\def\PYG@tc##1{\textcolor[rgb]{0.73,0.38,0.84}{##1}}}
\expandafter\def\csname PYG@tok@mf\endcsname{\def\PYG@tc##1{\textcolor[rgb]{0.13,0.50,0.31}{##1}}}
\expandafter\def\csname PYG@tok@vg\endcsname{\def\PYG@tc##1{\textcolor[rgb]{0.73,0.38,0.84}{##1}}}
\expandafter\def\csname PYG@tok@nd\endcsname{\let\PYG@bf=\textbf\def\PYG@tc##1{\textcolor[rgb]{0.33,0.33,0.33}{##1}}}
\expandafter\def\csname PYG@tok@ch\endcsname{\let\PYG@it=\textit\def\PYG@tc##1{\textcolor[rgb]{0.25,0.50,0.56}{##1}}}
\expandafter\def\csname PYG@tok@s2\endcsname{\def\PYG@tc##1{\textcolor[rgb]{0.25,0.44,0.63}{##1}}}
\expandafter\def\csname PYG@tok@nf\endcsname{\def\PYG@tc##1{\textcolor[rgb]{0.02,0.16,0.49}{##1}}}
\expandafter\def\csname PYG@tok@kp\endcsname{\def\PYG@tc##1{\textcolor[rgb]{0.00,0.44,0.13}{##1}}}
\expandafter\def\csname PYG@tok@nt\endcsname{\let\PYG@bf=\textbf\def\PYG@tc##1{\textcolor[rgb]{0.02,0.16,0.45}{##1}}}
\expandafter\def\csname PYG@tok@s\endcsname{\def\PYG@tc##1{\textcolor[rgb]{0.25,0.44,0.63}{##1}}}
\expandafter\def\csname PYG@tok@gu\endcsname{\let\PYG@bf=\textbf\def\PYG@tc##1{\textcolor[rgb]{0.50,0.00,0.50}{##1}}}
\expandafter\def\csname PYG@tok@sx\endcsname{\def\PYG@tc##1{\textcolor[rgb]{0.78,0.36,0.04}{##1}}}
\expandafter\def\csname PYG@tok@sh\endcsname{\def\PYG@tc##1{\textcolor[rgb]{0.25,0.44,0.63}{##1}}}
\expandafter\def\csname PYG@tok@na\endcsname{\def\PYG@tc##1{\textcolor[rgb]{0.25,0.44,0.63}{##1}}}
\expandafter\def\csname PYG@tok@ge\endcsname{\let\PYG@it=\textit}
\expandafter\def\csname PYG@tok@cs\endcsname{\def\PYG@tc##1{\textcolor[rgb]{0.25,0.50,0.56}{##1}}\def\PYG@bc##1{\setlength{\fboxsep}{0pt}\colorbox[rgb]{1.00,0.94,0.94}{\strut ##1}}}
\expandafter\def\csname PYG@tok@gp\endcsname{\let\PYG@bf=\textbf\def\PYG@tc##1{\textcolor[rgb]{0.78,0.36,0.04}{##1}}}
\expandafter\def\csname PYG@tok@si\endcsname{\let\PYG@it=\textit\def\PYG@tc##1{\textcolor[rgb]{0.44,0.63,0.82}{##1}}}
\expandafter\def\csname PYG@tok@ss\endcsname{\def\PYG@tc##1{\textcolor[rgb]{0.32,0.47,0.09}{##1}}}
\expandafter\def\csname PYG@tok@gs\endcsname{\let\PYG@bf=\textbf}
\expandafter\def\csname PYG@tok@cpf\endcsname{\let\PYG@it=\textit\def\PYG@tc##1{\textcolor[rgb]{0.25,0.50,0.56}{##1}}}
\expandafter\def\csname PYG@tok@k\endcsname{\let\PYG@bf=\textbf\def\PYG@tc##1{\textcolor[rgb]{0.00,0.44,0.13}{##1}}}

\def\PYGZbs{\char`\\}
\def\PYGZus{\char`\_}
\def\PYGZob{\char`\{}
\def\PYGZcb{\char`\}}
\def\PYGZca{\char`\^}
\def\PYGZam{\char`\&}
\def\PYGZlt{\char`\<}
\def\PYGZgt{\char`\>}
\def\PYGZsh{\char`\#}
\def\PYGZpc{\char`\%}
\def\PYGZdl{\char`\$}
\def\PYGZhy{\char`\-}
\def\PYGZsq{\char`\'}
\def\PYGZdq{\char`\"}
\def\PYGZti{\char`\~}
% for compatibility with earlier versions
\def\PYGZat{@}
\def\PYGZlb{[}
\def\PYGZrb{]}
\makeatother

\renewcommand\PYGZsq{\textquotesingle}

\begin{document}

\maketitle
\tableofcontents
\phantomsection\label{index::doc}


Contents:


\chapter{カルマンフィルタとは?}
\label{docs/about_kalman::doc}\label{docs/about_kalman:id1}\label{docs/about_kalman:kalmanfilter}
カルマンフィルタとは、観測されたデータに基づき、線形確率システムの状態を逐次的に推定するアルゴリズムです。
時系列データならばどのようなデータに対して適用することができ、航空工学、ロボット工学、画像処理、計量経済学、農学など
多岐にわたる分野において適用例があります。


\chapter{何ができるの?}
\label{docs/about_kalman:id2}
カルマンフィルタが適用される場面は主にノイズ除去・未来予測・外れ値推定・センサヒュージョンです。
カルマンフィルタは線形・雑音がガウスノイズという性質を持ち、そのため一般の状態空間モデルと比較して少ない計算量
であるという性質を持ちます。


\section{LQGの仮定}
\label{docs/about_kalman:lqg}\begin{enumerate}
\item {} 
システム方程式及び観測方程式の線形性 (Linear)

\item {} 
システムの雑音及び観測雑音の白色性 (white)

\item {} 
雑音分布のガウス正規性 (Gaussian)

\end{enumerate}

4. 2乗誤差規範
カルマンフィルタはLQG仮定の元、事後推定誤差分散を最小化することを目的としている


\section{確率システム}
\label{docs/about_kalman:id3}
システム方程式

\(p(z_{n}|z_{n-1})=N(z_{n}|Az_{n-1},\Gamma)\)

\(p(x_{n}|z_{n})=N(x_{n}|Cz_{n},\Sigma)\)


\section{定式化}
\label{docs/about_kalman:id4}
観測更新ステップ

\(p(z_{t}|X^{t})=\frac{p(x_{t}|z_{t})p(z_{t}|X^{t-1})}{p(x_{t}|X^{t-1})}\)

時間更新ステップ

\(p(z_{t+1}|X^{t})=\int p(z_{t+1}|z_{t})p(z_{t}|X^{t}) dz_{t}\)


\section{用語と記号}
\label{docs/about_kalman:id5}
事前推定誤差: \(\hat{e} = \mu_{k} - \bar{\mu}_{k} = \mu_{k} - E[z|x_{k-1}]\)

事後推定誤差: \(\hat{e} = \mu_{k} - \hat{\mu}_{k} = \mu_{k} - E[z|x_{k}]\)

事前推定誤差分散: \(P_{k} = E[\hat{e}_{k} \hat{e}_{k}^{T}]\)

事後推定誤差分散: \(P_{k} = E[e_{k} e_{k}^{T}]\)


\section{計算}
\label{docs/about_kalman:id6}

\subsection{予測}
\label{docs/about_kalman:id7}\begin{itemize}
\item {} 
事前平均値

\(\mu_{n}^{-} = A \mu_{n-1}\)

\item {} 
事前分散

\(P_{n-1}=A V_{n-1} A^{T} + \Gamma\)

\end{itemize}


\subsection{フィルタリング}
\label{docs/about_kalman:id8}\begin{itemize}
\item {} 
カルマンゲイン

\(K_{n}=P_{n-1}C^{T}(CP_{n-1}C^{T}+\Sigma)^{-1}\)

\item {} 
事後平均値

\(\mu_{n}=\mu_{n}^{-} +K_{n}(x_{n}-C \mu_{n}^{-})\)

\item {} 
事後分散

\(V_{n}=(I-K_{n}C)P_{n-1}\)

\end{itemize}


\section{メモ}
\label{docs/about_kalman:id9}\begin{itemize}
\item {} 
システムが定常であるとき、カルマンゲインは収束する
\begin{quote}

\(\mu_{n}=A\mu_{n-1}+K_{n}(x_{n}-CA\mu_{n-1})= A\mu_{n-1}+\frac{P_{n-1}}{P_{n-1}+\frac{\Sigma}{C^2}}\frac{(x_{n}-\hat{x}_{n})}{C}\)
\end{quote}

\item {} 
\(\Sigma << P_{n-1}\) の時、すなわち、システム方程式の分散が、観測値の分散よりも大きい場合、カルマンゲインは大きくなり、観測に基づく推定結果が支配的になる。

\end{itemize}


\section{参考}
\label{docs/about_kalman:id10}\begin{itemize}
\item {} 
非線形カルマンフィルタ

\item {} 
PRML 下巻

\end{itemize}


\chapter{パラメータ推定}
\label{docs/estimate_parameter::doc}\label{docs/estimate_parameter:id1}

\section{0. カルマンフィルタの尤度関数}
\label{docs/estimate_parameter:id2}
完全データX,Zが

\(l(\theta) = \ln p(X,Z|\theta) = \ln p(z_{0}|\mu_{0},P_{0})+ \sum_{n=2}^{N} \ln p(z_{n}|z_{n-1},F,Q)+ \sum_{n=1}^{N} \ln p(x_{n}|z_{n},H,R)\)

\(\ln \mathcal{N} (\mathbf{x}|\mathbf{\mu},\mathbf{\Sigma}) = - \frac{D\ln(2 \pi)}{2}  -\frac{1}{2} \ln(|\Sigma|) -\frac{1}{2}(\mathbf{x}- \mathbf{\mu}) \Sigma^{-1}(\mathbf{x}- \mathbf{\mu})\)


\section{1. EMアルゴリズムを用いて求める}
\label{docs/estimate_parameter:em}
パラメータ \(\theta = \lbrace A , \Gamma , C , \Sigma , \mu_{0} , P_{0} \rbrace\) をEMアルゴリズムによって求める
\begin{itemize}
\item {} 
完全データの対数尤度関数

\(\ln p(\mathbf{X},\mathbf{Z}|\theta) = \ln(z_{1}|\mu_{0},P_{0}) + \sum_{n=2}^N \ln p(z_{n}|z_{n-1},A,\Gamma) + \sum_{n=1}^{N} \ln p(x_{n}|z_{n},C,\Sigma)\)

\item {} 
完全データの尤度関数の期待値

\(Q(\theta,\theta^{old})=E_{\mathbf{Z}|\theta^{old}}[\ln p(\mathbf{X},\mathbf{Z}|\theta)] = \sum_{\mathbf{Z}} \ln p(\mathbf{X},\mathbf{Z}|\theta) p(\mathbf{Z}|\mathbf{X},\theta^{old})\)

\end{itemize}


\section{E-step}
\label{docs/estimate_parameter:e-step}
E-stepにおける隠れ変数の分布はカルマンスムーザーの方法を用いる
\begin{itemize}
\item {} 
\(p(\mathbf{Z}|\mathbf{X},\theta^{old})\)

\item {} 
\(J_{n} = V_{n} A^{T} (P_{n})^{-1}\)

\item {} 
\(\hat{\mu}_{n} = \mu_{n} + J_{n}(\hat{\mu}_{n+1} - A \mu_{n})\)

\item {} 
\(\hat{V}_{n} = V_{n} + J_{n}(\hat{V}_{n+1} -P_{n})J_{n}^{T}\)

\item {} 
\(E[z_{n}] = \hat{\mu_{n}}\)

\item {} 
\(E[z_{n} z_{n-1}^{T}] = \hat{V}_{n} J_{n-1}^{T} + \hat{\mu}_{n} \hat{\mu}_{n-1}^{T}\)

\item {} 
\(E[z_{n} z_{n}^{T}] = \hat{V}_{n} + \hat{\mu}_{n} \hat{\mu}_{n}^{T}\)

\end{itemize}


\section{M-step}
\label{docs/estimate_parameter:m-step}\begin{itemize}
\item {} 
\(\mu_{0}^{new} = E[z_{1}]\)

\item {} 
\(P_{0}^{new} = E[z_{1} z_{1}^{T}] - E[z_{1}]E[z_{1}^{T}]\)

\item {} 
\(A^{new} =(\sum_{n=2}^{N} E[z_{n} z_{n-1}^{T}])(\sum_{n=2}^{N} E[z_{n-1} z_{n-1}^{T}])^{-1}\)

\item {} 
\(\Gamma^{new} = \frac{1}{N-1} \sum_{n=2}^{N}(E[z_{n}z_{n}^{T}] - A ^{new} E[z_{n-1}z_{n}^{T}]-E[z_{n} z_{n-1}^{T}](A^{new})^{T} + A^{new} E[z_{n-1} z_{n-1}^{T}](A^{new})^{T})\)

\item {} 
\(C^{new} = (\sum_{n=1}^{N}x_{n}E[z_{n}^{T}])(\sum_{n=1}^{N}E[z_{n}z_{n}^{T}])^{-1}\)

\item {} 
\(\Sigma^{new} = \frac{1}{N} \sum_{n=1}^{N} (x_{n}x_{n}^{T} -C^{new} E[z_{n}]x_{n}^{T} -x_{n}E[z_{n}^{T}](C^{new})^{T}+C^{new}E[z_{n}z_{n}^{T}](C^{new})^{T})\)

\end{itemize}


\section{2. MCMC法により求める}
\label{docs/estimate_parameter:mcmc}

\chapter{Indices and tables}
\label{index:indices-and-tables}\begin{itemize}
\item {} 
\DUspan{xref,std,std-ref}{genindex}

\item {} 
\DUspan{xref,std,std-ref}{modindex}

\item {} 
\DUspan{xref,std,std-ref}{search}

\end{itemize}



\renewcommand{\indexname}{索引}
\printindex
\end{document}
